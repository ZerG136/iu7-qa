\rc{Аналитическая геометрия}{2}

\section{Теоретические вопросы}

\subsection{Теоретические вопросы. Базовый уровень}

\begin{question}
  Дать определение единичной, нулевой, верхней треугольной и нижней треугольной матрицы.
\end{question}
\begin{answer}
  \textit{Нулевой матрицей} называется матрица, все элементы которой равные нулю. \[
  \theta = \begin{pmatrix}
    0 & 0 & 0 \\
    0 & 0 & 0
  \end{pmatrix}
\]
  \textit{Верхне-треугольной матрицей} называется квадратная матрица, у которой элементы под главной диагональю равны нулю. \[
  C = \begin{pmatrix}
    1 & 2 & 3 \\
    0 & 5 & 6 \\
    0 & 0 & 9
  \end{pmatrix}
  \] 
  \textit{Нижне-треугольной матрицей} называется квадратная матрица, у которой над главной диагональю равны нулю. \[
  D = \begin{pmatrix}
    1 & 0 & 0 \\
    4 & 5 & 0 \\
    7 & 8 & 9
  \end{pmatrix}
\]
\end{answer} 

\begin{question}
  Дать определение равенства матриц.
\end{question} 
\begin{answer}
  Две матрицы называются \textit{равными}, если они имеют одинаковую размерность, и их соответствующие элементы равны. 
\end{answer} 

\begin{question}
  Дать определение суммы матриц и произведения матрицы на число.
\end{question} 
\begin{answer}
  \textit{Суммой матриц} $A_{m \times n}$ и $B_{m \times n}$ называется матрица $C_{m \times n}$, элементы которой являются суммой соответствующих элементов матриц $A$ и $B$. 

  \textit{Произведением матрицы} $A_{m \times n}$ на число $k = const$ называется матрица $C_{m \times n}$, элементы которой равны произведению соответствующего элемента матрицы на данное число $c_{ij} = k a_{ij}$.
\end{answer} 

\begin{question}
  Дать определение операции транспонирования матриц.
\end{question} 
\begin{answer}
  \textit{Транспонированной матрицей} $A_{mn}$ называется матрица размерностью $n \times  m$, элементы которой:
  \begin{gather*}
    a^\tau_{ij} = a_{ji} \\
    A_{n \times m}^\tau \text{ -- транспонированная матрица } A_{m \times n}
  \end{gather*}
\end{answer} 

\begin{question}
  Дать определение операции умножения матриц.
\end{question} 
\begin{answer}
    \textit{Произведением матриц} $A$ и $B$ назвается матрица $C$, элементы которой определяются как: \[
    c_{ij} = \sum_{l=1}^{k} a_{il} \cdot b_{lj}
  \] 
\end{answer} 

\begin{question}
  Дать определение обратной матрицы.
\end{question} 
\begin{answer}
  \textit{Обратная матрица} квадратной матрицы $A_{n \times n}$ называется матрица $A^{-1}_{n \times n}$ такая, что $A \times A^{-1} = A^{-1} \times A = E$.
\end{answer} 

\begin{question}
  Дать определение минора. Какие миноры называются окаймляющими для данного минора матрицы?
\end{question} 
\begin{answer}
  \textit{Минором k-ого порядка} матрицы $A$ называется определитель, составленный из пересечения $k$ строк и $k$ столбцов с сохранением их порядка.
\end{answer} 

\begin{question}
  Дать определение базисного минора и ранга матрицы.
\end{question} 
\begin{answer}
  \textit{Базисным минором} называется матрицы $A$ называется минор, не равный нулю, порядок которого равен рангу матрицы $A$.

  \textit{Рангом матрицы} называется число $A$, равное наибольшему порядку, отличному от нуля, минора матрицы $A$.
\end{answer} 

\begin{question}
  Дать определение однородной и неоднородной СЛАУ.
\end{question} 
\begin{answer}
  СЛАУ, у которой все свободные члены равны нулю, называется \textit{однородной}.

  СЛАУ, у которой хотя бы один свободный член не равен нулю, называется \textit{неоднородной}.
\end{answer} 

\begin{question}
  Дать определение фундаментальной системы решений однородной СЛАУ.
\end{question} 
\begin{answer}
  Набор решений однородной СЛАУ называется \textit{фундаментальной системой решений (ФСР)}.\[
  k = n - r \quad r = Rg(A), n \text{ -- кол-во неизвестных в СЛАУ}
  \] 
\end{answer} 

\begin{question}
  Записать формулы для нахождения обратной матрицы к произведению двух обратимых матриц и для транспонированной матрицы.
\end{question} 
\begin{answer}
  Что? Вот Вы поняли в чём вопрос? А я нет. Куда писать, вы знаете :)
\end{answer} 

\begin{question}
  Дать определение присоединённой матрицы и записать формулу для вычисления обратной матрицы.
\end{question} 
\begin{answer}
  Матрица $A^*$, являющаяся транспонированной матрицей алгебраических дополнений матрицы A называется  \textit{присоединённой матрицей}.
  \begin{gather*}
    A^{-1} = \frac{1}{det A} \cdot 
    \begin{pmatrix}
      A_{11} & A_{12} & \ldots & A_{1n} \\
      A_{21} & A_{22} & \ldots & A_{2n} \\
      \ldots & \ldots & \ldots & \ldots \\
      A_{n1} & A_{n2} & \ldots & A_{nn} \\
    \end{pmatrix}
  \end{gather*}
\end{answer} 

\begin{question}
  Перечислить элементарные преобразования матриц.
\end{question} 
\begin{answer}.
  \begin{itemize}
    \item Перемена мест двух строк или двух столбцов в данной матрице;
    \item Умножение строки (или столбца) на произвольное число, отличное от нуля;
    \item Прибавление к одной строке (столбцу) другой строки (столбца), умноженной на некоторое число.
  \end{itemize}
\end{answer} 

\begin{question}
  Записать формулы Крамера для решения системы линейных уравнений с обратимой матрицей.
\end{question} 
\begin{answer}
  Для СЛАУ:
  \begin{align*}
    \begin{cases}  
      a_{11}x_1 + a_{12}x_2 + \ldots + a_{1n} &= b_1 \\
      a_{21}x_1 + a_{22}x_2 + \ldots + a_{2n} &= b_2 \\
      \ldots
      a_{n1}x_1 + a_{n2}x_2 + \ldots + a_{nn} &= b_n \\
    \end{cases}
  \end{align*}
  Значение $x_i$ ($i=1..n$) вычисляется по формуле:
  \begin{gather*}
    x_i = \frac{1}{\Delta} 
    \begin{vmatrix}
      a_{11} & \ldots & a_{1,i-1} & b_1 & a_{1,i+1} & \ldots & a_{1n} \\ 
      a_{21} & \ldots & a_{2,i-1} & b_2 & a_{2,i+1} & \ldots & a_{2n} \\ 
      \ldots & \ldots & \ldots & \ldots & \ldots & \ldots & \ldots \\
      a_{n-1,1} & \ldots & a_{n-1,i-1} & b_{n-1} & a_{n-1,i+1} & \ldots & a_{n-1,n} \\ 
      a_{n,1} & \ldots & a_{n,i-1} & b_{n} & a_{n,i+1} & \ldots & a_{n,n}
    \end{vmatrix}
  \end{gather*}
\end{answer} 

\begin{question}
  Перечислить различные формы записи системы линейных алгебраических уравнений (СЛАУ). Какая СЛАУ называется совместной?
\end{question}
\begin{answer}
  Формы записи СЛАУ:
  \begin{enumerate}
    \item Координатная
    \item Матричная
    \item Векторная
  \end{enumerate}
  СЛАУ, имеющая решение, назыается \textit{совместной}.
\end{answer} 

\begin{question}
  Привести пример, показывающий, что умножение матриц некоммутативно.
\end{question} 
\begin{answer}
   \textit{От автора: просто возьмите две случайные матрицы и перемножьте их в одну и в другю сторону. Достаточно 2x2}. 
\end{answer} 

\begin{question}
  Сформулировать свойства ассоциативности умножения матриц и дистрибутивности умножения относительно сложения.
\end{question} 
\begin{answer}
  \begin{itemize}.
    \item Ассоциативность \[
      (A \times B) \times C = A \times (B \times C)
    \]
    \item Дистрибутивность произведения матриц относительно сложения: \[
      (A + B) \times C = A \times C + B \times C
    \] 
  \end{itemize} 
\end{answer} 

\begin{question}
  Сформулировать критерий Кронекера — Капелли совместности СЛАУ.
\end{question} 
\begin{answer}
  Для того, чтобы СЛАУ была совместной, необходимо и достаточно, чтобы ранг матрицы $A$ был равен рангу матрицы A|B.
\end{answer} 

\begin{question}
  Сформулировать теорему о базисном миноре.
\end{question} 
\begin{answer}
  Строки (столбцы) матрицы $A$, входящие в базисный минор -- базисы. \\
  Базисные строки (столбцы), входящие в базисный минор -- линейно-независимы. \\
  Любую строку (столбец), не входящую в базисный минор, можно представить в виде линейной комбинации базисных строк (столбцов).
\end{answer} 

\begin{question}
  Сформулировать теорему о свойствах решений однородной СЛАУ.
\end{question} 
\begin{answer}
  Пусть $X^{(0)}, X^{(1)}, \ldots X^{(n)}$ -- решения однородных СЛАУ $A \times X = \theta$. Тогда их линейной комбинацией так же является решением однородной СЛАУ.
\end{answer} 

\begin{question}
  Сформулировать теорему о структуре общего решения неоднородной СЛАУ.
\end{question} 
\begin{answer}
  
\end{answer} 

\begin{question}
  Сформулировать теорему о структуре общего решения однородной СЛАУ.
\end{question} 
\begin{answer}
  Пусть $X^{(0)}$ -- некоторое частное решение неоднородной СЛАУ  $A \times X = B$.
  Пусть  $X^{(1)} \ldots$ -- некоторая ФСР, соответствующая однородной СЛАУ $A \times X=\theta$.
  Тогда общее решение неоднородной СЛАУ $A \times X = B$ будет иметь вид :  \[
    X_o = X^{(0)} + c_1 X^{(1)} + c_2 X^{(2)} + \ldots + c_n X^{(n)}, \quad c_i = const
  \] 
\end{answer} 

\begin{question}
  Сформулировать теорему об инвариантности ранга при элементарных преобразованиях матрицы.
\end{question} 
\begin{answer}
  Ранг матрицы не меняется при элементарных преобразований строк (столбцов) матрицы. Ганг матрицы равен количеству ненулевых строк (столбцов) ступенчатой матрицы, полученной путём элементарных преобразований.
\end{answer} 

\begin{question}
  Сформулировать критерий существования обратной матрицы.
\end{question} 
\begin{answer}
  Для того, чтобы матрица A имела обратную необходимо и достаточно, чтобы её определитель не равнялся нулю. 
\end{answer} 

\subsection{Теоретические вопросы. Повышенная сложность}

\begin{question}
  Доказать теорему о связи решений неоднородной и соответствующей однородной СЛАУ и теорему о структуре общего решения неоднородной СЛАУ.
\end{question} 
\begin{answer}
  \textit{О связи решений неоднородной и соответствующей однородной СЛАУ}. \\
  Пусть $X^{(0)}$ -- некоторое решение неоднордной СЛАУ $A \times X = B$. Произвольный стобец $X$ является решением СЛАУ $A \times X = B$ тогда и только тогда, когда его можно представить в виде:  \[
    X = X^{(0)} + Y, \quad \text{ где } A \times Y = \theta
  \] 
  Доказательство:

  1) Необходимость. \\
  Пусть $X$ -- решение СЛАУ $A \times X = B$. Обозначим $Y = X - X^{(0)}$
  \begin{gather*}
    A \times Y = A \times (X - X^{(0)}) = A \times X - A \times X^{(0)} = B - B = \theta
  \end{gather*}
  Значит $Y$ является решением соответствующей однородной СЛАУ  $A \times Y = \theta$.
  
  2) Достаточность. \\
  Пусть $X$ можно представить в виде:  \[
    X = X^{(0)} + Y, \quad \text{ где } A \times Y = 0
  \]
  Тогда:
  \begin{gather*}
  A \times X = A \times (X^{(0)} + Y) = A \times X^{(0) + A \times Y} = B + \theta = B
  \end{gather*}
  Отсюда делаем вывод, что $X$ является решением неоднородной СЛАУ.
\end{answer} 

\begin{question}
  Доказать теорему о базисном миноре.
\end{question} 
\begin{answer}
  \textit{ Строки (столбцы) матрицы $A$, входящие в базисный минор -- базисы. \\
  Базисные строки (столбцы), входящие в базисный минор -- линейно-независимы. \\
Любую строку (столбец), не входящую в базисный минор, можно представить в виде линейной комбинации базисных строк (столбцов).}

  Доказательство: 

  Пусть ранг матрицы A равен R. \\
  Предположим, что строки матрицы $A$ - линейно-зависимы. 
  Тогда одну из ни можно выразить как линейную комбинацию других строк. 
  Тогда в базисном миноре 1-ая строка -- линейная комбинация других строк. 
  По свойству определителей этот минор равен нулю, что противоречит определению базисного минора. \\
  \\
  Пусть базисный минор состоит из первых $r$ строк и $r$ столбцов матрицы $A$. 
  Добавим к этому минору произвольную i-ную строку и j-ный столбец -- получим окаймляющий минор. 
  Если $j \le r$, то в миноре $M'$ 2 одинаковых столбца и минор равен нулю.
  Если $j > r$, то в минор  $M'$ тоже равен нулю, т.к. ранг матрицы A равен r, наибольний порядок, отличный от нуля, минора равен $j$. \\
  \\
  Определитель можно вычислить путём разложения по каой-нибудь строке или столбцу, поэтому найдем определитель $M'$ путём разложения по j-ному столбцу:
   \begin{gather*}
     a_{1j} A_{1j} + a_{2j} A_{2j} + \ldots + a_{ij} A_{ij} = 0 \\
     j = r + 1 \implies \\
     a_{1r+1} A_{1r+1} + a_{2r+1} A_{2r+1} + \ldots + a_{ir+1} A_{ir+1} = 0 \\
  \end{gather*}
  $A_{r+1,r+1}$ -- базисный минор, т.к. $M \neq 0$, то $A_{r+1,r+1} \neq 0$.
  \begin{gather*}
    a_{r+1,r+1} =
    - \frac{A_{1,r+1}}{A_{r+1,r+1}} \cdot  a_{1,r+1}
    - \frac{A_{2,r+1}}{A_{r+1,r+1}} \cdot  a_{2,r+1}
    \ldots
    - \frac{A_{r,r+1}}{A_{r+1,r+1}} \cdot  a_{r,r+1}
  \end{gather*}
  Обозначим $\lambda_i = -\frac{A_{i,r+1}}{A_{r+1}{r+1}}$
  \begin{gather*}
    a_{r+1,r+1} = \lambda_i a_{1,r+1} + \lambda_2 a_{2,r+1} + \ldots + \lambda_r \cdot a_{r,r+1} \\
  \end{gather*}
  Элементы $i$-ой строки можно представить в виде линейной комбинации строк.
\end{answer} 

\begin{question}
  Доказать свойства ассоциативности и дистрибутивности умножения матриц.
\end{question} 
\begin{answer}
  Здесь могло быть Ваше доказательство\ldots
\end{answer} 

\begin{question}
  Доказать критерий существования обратной матрицы.
\end{question} 
\begin{answer}
  \textit{Для того, чтобы матрица $A$ имела обратную необходимо и достаточно, чтобы её определитель не равнялся нулю.}

  Доказательство:

  1) Пусть матрица A имеет обратную, тогда по определению: \[
    A \times A^{-1} = E
  \]
  В таком случае:
  \begin{gather*}
    det(A \times A^{-1}) = det(E) = 1 \\
    det(A \times A^{-1}) = det(A) \cdot det(A^{-1}) = 1 \implies det A \neq 0
  \end{gather*}

  2) Пусть $det A \neq 0$. Если матрицн разложить по строке или столбцу:
  \begin{align*}
    &\sum_{j+1}^{n} a_{ij} A_{ij} = a_{i1} A_{i1} + a_{i2} A_{i2} + \ldots + a_{in} A_{in} = det A \\
    &\sum_{j+1}^{n} a_{ij} A_{nj} = a_{i1} A_{n1} + a_{i2} A_{n2} + \ldots + a_{in} A_{nn} = 0 \quad i \neq k
  \end{align*}

  Пусть существует матрица $B$: \[
    b_{ij} = \frac{A_{ij}}{det A}
  \] 
  Пусть $C = A \cdot B$:
  \begin{gather*}
    c_{ij} = \sum_{k=1}^{n} a_{ik} \cdot b_{in} = \sum_{n=1}^{n} a_{ik} \frac{A_{jn}}{det A} \\
    = \frac{1}{det A} \cdot \sum_{n=1}^{n} a_{ik} A_{jn} = \begin{cases}
      \frac{1}{det A} \cdot det A = 1, \quad \text{ если } i=j \\
      \frac{1}{det A} \cdot 0 = 0, \quad \text{ если } i \neq j
    \end{cases} \implies C = E \\
    c_{ij} = 1, \text{ если } i=j \\
    c_{ij} = 0, \text{ если } i \neq j
  \end{gather*}

  Получим: 
  \begin{gather*}
    \begin{rcases}
      A \times B = E \\
      B \times A = E
    \end{rcases} \implies \text{ по определению } B = A^{-1}
  \end{gather*}
\end{answer} 

\begin{question}
  Доказать критерий Кронекера — Капелли совместности СЛАУ.
\end{question} 
\begin{answer}
  \textit{Для того, чтобы СЛАУ была совместной, необходимо и достаточно, чтобы ранг матрицы $A$ был равен рангу матрицы A|B.}
  
  У кого есть полное доказательство критерия -- Вы знаете, куда писать, чтобы оно оказалось здесь :).
\end{answer} 

\begin{question}
  Доказать теорему о существовании ФСР однородной СЛАУ.
\end{question}
\begin{answer}
  \textit{Пусть имеется однородная СЛАУ $A \times X = \theta$ с $n$ неизвестных и $rg(A) = r$. \\
  Тогда существует набор  $k = n - r$ решений однородной СЛАУ, которые образуют ФСР:  \[
    X^{(1)}, X^{(2)}, \ldots X^{(k)}
\]}

  Доказательство:

  Пусть базисный минор $M$ матрицы A состоит из первых $r$ строк и первых $r$ столбцов матрицы $A$.
  Тогда любая строка $A$, от $r+1$ до $m$ будет линейной комбинацией строк базисного минора.

  Если  $x_1, x_2, \ldots x_{n}$ удовлетворяют уравнениям СЛАУ соответветствующим строкам базисного минора то это решение будет удовлетворять и остальным уравнениям СЛАУ.
  Поэтому исключим из системы уравнения после $r$-ой строки:
  \begin{gather*}
    \begin{cases}
      a_{11} x_1 + a_{12} x_2 + \ldots + a_{1r} x_r + \ldots + a_{1n} x_{n} = 0 \\
      a_{21} x_1 + a_{22} x_2 + \ldots + a_{2r} x_r + \ldots + a_{2n} x_{n} = 0 \\
      \ldots \\
      a_{r1} x_1 + a_{r2} x_2 + \ldots + a_{rr} x_r + \ldots + a_{rn} x_{n} = 0 \\
    \end{cases} \tag{3} 
  \end{gather*}

  Переменные, соответствующие базисным столбцам, называют \textit{базисными}, остальные -- \textit{свободными}.

  В системе (3) бащисными переменными являются переменные $x_1, x_2, \ldots x_r$; свободными являются переменные $x_{r+1}, x_{r+2}, \ldots x_n$.

  Оставим в левой части слагаемые с базисными переменными, а в правой -- со свободными:
  \begin{gather*}
    \begin{cases}
      a_{11} x_1 + a_{12} x_2 + \ldots a_{1r} x_r = a_{1,r+1} x_{r+1} + \ldots + a_{1n} x_{n} \\ 
      a_{21} x_1 + a_{22} x_2 + \ldots a_{2r} x_r = a_{2,r+1} x_{r+1} + \ldots + a_{2n} x_{n} \\ 
      \ldots \\
      a_{r1} x_1 + a_{r2} x_2 + \ldots a_{rr} x_r = a_{r,r+1} x_{r+1} + \ldots + a_{rn} x_{n} \\ 
    \end{cases} \tag{4} 
  \end{gather*}

  Если свободным переменным придавать различные значения, то определитель левой части (4) равен базисному минору $A (\neq 0)$, то (4) будет иметь единственное решение.

  Возьмём $k$ наборов свободных переменных:
   \begin{gather*}
    \begin{matrix}
      X^{(1)}_{r+1} & X^{(2)}_{r+1} & \ldots & X^{(k)}_{r+1} \\
      X^{(1)}_{r+2} & X^{(2)}_{r+2} & \ldots & X^{(k)}_{r+2} \\
      \ldots & \ldots & \ldots & \ldots \\
      X^{(1)}_n = 0 & X^{(2)}_n = 0 & \ldots & X^{(k)}_n = 0 & 
    \end{matrix}
  \end{gather*}

  В результате, при каждом наборе свободных переменных мы получаем $k$ решений однородной СЛАУ:
   \begin{gather*}
     X^{(i)} = 
     \begin{pmatrix}
       X^{(i)}_1 \\
       X^{(i)}_2 \\
       \ldots \\
       X^{(i)}_r \\
       \ldots \\
       X^{(i)}_n
     \end{pmatrix}
  \end{gather*}

  Пусть линейная комбинация решений равна 0:
  \begin{gather*}
    \lambda_1
   \begin{pmatrix}
       X^{(1)}_1 \\
       X^{(1)}_2 \\
       \ldots \\
       X^{(1)}_r \\
       \ldots \\
       X^{(1)}_n
     \end{pmatrix}
     + \lambda_2
   \begin{pmatrix}
       X^{(2)}_1 \\
       X^{(2)}_2 \\
       \ldots \\
       X^{(2)}_r \\
       \ldots \\
       X^{(2)}_n
    \end{pmatrix}
    + \ldots + \lambda_k
   \begin{pmatrix}
       X^{(k)}_1 \\
       X^{(k)}_2 \\
       \ldots \\
       X^{(k)}_r \\
       \ldots \\
       X^{(k)}_n
    \end{pmatrix}
  = 
  \begin{pmatrix}
    0 \\ 0 \\ \ldots \\ 0 \\ \ldots \\ 0
  \end{pmatrix} = \theta
  \end{gather*}
  
  \begin{align*}
    r+1: \quad &1 \cdot\lambda_1 + 0 \cdot \lambda_2 + \ldots + 0 \cdot \lambda_k = 0 \implies \lambda_1 = 0 \\
    r+2: \quad &0 \cdot\lambda_1 + 1 \cdot \lambda_2 + \ldots + 0 \cdot \lambda_k = 0 \implies \lambda_2 = 0 \\
          &\ldots \\
     r: \quad &1 \lambda_1 + 0 \cdot \lambda_2 + \ldots + 1 \cdot \lambda_k = 0 \implies \lambda_k = 0
  \end{align*}
  
  Все коеффициенты равны нулю. Мы получили тривиальную равную нуля линейную комбинация решений однородной СЛАУ.
\end{answer} 

\begin{question}
  Вывести формулы Крамера для решения системы линейных уравнений с обратимой матрицей.
\end{question} 
\begin{answer}
  Запишем СЛАУ в матричном виде: 
\begin{gather*}
  A \times X = B \qquad A_{n \times n} \\
  A = \left( 
  \begin{matrix}
    a_{11} & a_{12} & \ldots & a_{1n} \\
    a_{21} & a_{22} & \ldots & a_{2n} \\
    \ldots & \ldots & \ldots & \ldots \\
    a_{n1} & a_{n2} & \ldots & a_{nn}
  \end{matrix}
  \right) 
\end{gather*}

Пусть матрица не вырожденная. Тогда её обратная матрица будет иметь вид: 
\begin{gather*}
  A^{-1} = \frac{1}{det A} \left( 
    \begin{matrix}
    A_{11} & A_{12} & \ldots & A_{1n} \\
    A_{21} & A_{22} & \ldots & A_{2n} \\
    \ldots & \ldots & \ldots & \ldots \\
    A_{n1} & A_{n2} & \ldots & A_{nn}
    \end{matrix}
  \right)^{\tau} = \left( 
  \begin{matrix}
    \frac{A_{11}}{det A} & \frac{A_{12}}{det A} & \ldots & \frac{A_{1n}}{det A} \\
    \frac{A_{21}}{det A} & \frac{A_{22}}{det A} & \ldots & \frac{A_{2n}}{det A} \\
    \ldots & \ldots & \ldots & \ldots \\
    \frac{A_{n1}}{det A} & \frac{A_{n2}}{det A} & \ldots & \frac{A_{nn}}{det A}
  \end{matrix}
  \right) \\
  \\
  X = \left( 
  \begin{matrix}
    \frac{A_{11}}{det A} & \frac{A_{12}}{det A} & \ldots & \frac{A_{1n}}{det A} \\
    \frac{A_{21}}{det A} & \frac{A_{22}}{det A} & \ldots & \frac{A_{2n}}{det A} \\
    \ldots & \ldots & \ldots & \ldots \\
    \frac{A_{n1}}{det A} & \frac{A_{n2}}{det A} & \ldots & \frac{A_{nn}}{det A}
  \end{matrix}\right) \cdot \left( 
  \begin{matrix}
    b_1 \\
    b_2 \\
    \ldots \\
    b_{n}
  \end{matrix}
\right) \\
  \\
  x_i = 
  \frac{A_{i1}}{det A} b_1 +
  \frac{A_{i2}}{det A} b_2 +
  \ldots +
  \frac{A_{in}}{det A} b_n = \\
  = \frac{A_{i1}b_1 + A_{i_2}b_2 + \ldots + A_{in}b_n}{det A}
\end{gather*}

Заметим, что числитель последнего выражения это ничто иное, как:
\begin{gather*}
  x_i = \frac{1}{det A} 
    \begin{vmatrix}
      a_{11} & \ldots & a_{1,i-1} & b_1 & a_{1,i+1} & \ldots & a_{1n} \\ 
      a_{21} & \ldots & a_{2,i-1} & b_2 & a_{2,i+1} & \ldots & a_{2n} \\ 
      \ldots & \ldots & \ldots & \ldots & \ldots & \ldots & \ldots \\
      a_{n-1,1} & \ldots & a_{n-1,i-1} & b_{n-1} & a_{n-1,i+1} & \ldots & a_{n-1,n} \\ 
      a_{n,1} & \ldots & a_{n,i-1} & b_{n} & a_{n,i+1} & \ldots & a_{n,n}
    \end{vmatrix}
\end{gather*}
\end{answer} 

\begin{question}
  Доказать теорему о структуре общего решения однородной СЛАУ.
\end{question} 
\begin{answer}
  \textit{Пусть $X^{(1)}, X^{(2)}, \ldots X^{(k)}$ -- ФСР некоторой СЛАУ $A \times X = \theta$. Тогда общее решение однородной СЛАУ будет иметь вид: \[
    X = c_1 X^{(1)} + c_2 X^{(2)} + \ldots + c_k X^{(k)}, \quad c_i = const
\] }

  Доказательство:

  Пусть дана однородная СЛАУ:
  \begin{align*}
    \begin{cases}
      a_{11} x_1 + a_{12} x_2 + \ldots + a_{1n} x_{n} = 0 \\
      a_{21} x_1 + a_{22} x_2 + \ldots + a_{2n} x_{n} = 0 \\
      \ldots \\
      a_{m1} x_1 + a_{m2} x_2 + \ldots + a_{mn} x_{n} = 0
    \end{cases} \tag{1}
  \end{align*}
  Пусть $X = \left( \begin{matrix} x_1 \\ x_2 \\ \ldots \\ x_{n} \end{matrix} \right) $ -- решение СЛАУ, и матрица A имеет ранг $rg A = r$. 
  Тогда если $X$ является решением, то он ялвяется решением первых $r$ уравнений, соответствующих базисным строкам матрицы $A$.
  Пусть базисный минор стоит из первых $r$ строк и первых $r$ столбцов данной матрицы, тогда если $X$ -- решение уравнений с нулевого по $r$, то он является решением уравнений с  $r+1$ по $m$, которые являются линейной комбинацией первых  $k$ уравнений, поэтому уравнения с  $r+1$ по $m$ можно исключить.
  Т.к. базисный минор включает первые $r$ столбцов матрицы  $A$:
  \begin{gather*}
    M_r =
    \begin{pmatrix}
      a_{11} x_1 & a_{12} x_2 & \ldots & a_{1r} x_r \\
      a_{21} x_1 & a_{22} x_2 & \ldots & a_{2r} x_r \\
      \ldots & \ldots & \ldots & \ldots \\
      a_{r1} x_1 & a_{r2} x_2 & \ldots & a_{rr} x_r \\
      \ldots & \ldots & \ldots & \ldots \\
      a_{m1} x_1 & a_{m2} x_2 & \ldots & a_{mr} x_r \\
    \end{pmatrix}
  \end{gather*}
  то соответствующие этим столбцам переменные являются базисными (с $x_1$ по $x_{r}$), а остальные переменные (c $x_{r+1}$ по $ x_n$ ) -- свободными.

  После исключения первых $r$ строк, получаем:
  \begin{gather*}
    \begin{cases}
      a_{r1} x_1 + a_{r2} x_2 + \ldots + a_{rn} x_{n} = 0 \\
      a_{r+1,1} x_1 + a_{r+1,2} x_2 + \ldots + a_{r+1,n} x_{n} = 0 \\
      \ldots \\
      a_{m1} x_1 + a_{m2} x_2 + \ldots + a_{mr} x_{r} + \ldots + a_{mn} x_{n} = 0
    \end{cases} \tag{2}
  \end{gather*}

  Преобразуем уравнения так, что в левой части остались базисные переменные, а в правой -- свободные:
  \begin{gather*}
    \begin{cases}
      a_{11} x_1 + a_{12} + x_2 + \ldots + a_{1r} x_{r} = a_{1r+1} x_{r+1} + \ldots + a_{1n} x_{n} = 0 \\
      a_{21} x_1 + a_{22} + x_2 + \ldots + a_{2r} x_{r} = a_{2r+1} x_{r+1} + \ldots + a_{2n} x_{n} = 0 \\
    \ldots \\
      a_{m1} x_1 + a_{m2} + x_2 + \ldots + a_{mr} x_{r} = a_{mr+1} x_{r+1} + \ldots + a_{mn} x_{n} = 0 \tag{3}
    \end{cases}
  \end{gather*}
  Задавая различные значения свободных переменных, мы получаем, что система (3) будет иметь единственное решение, т.к. главный определитель данной системы будет равен угловому минору, не равному нулю.
  Решая эту систему получаем решение:
  \begin{gather*}
    \begin{cases}
      x_1 = x_{r+1} \lambda_{1,r+1} + x_{r+2} \lambda_{1,r+2} + \ldots + x_{n} \lambda_{1, n} \\
      x_2 = x_{r+1} \lambda_{2,r+1} + x_{r+2} \lambda_{2,r+2} + \ldots + x_{n} \lambda_{2, n} \\
      \ldots \\
      x_r = x_{r+1} \lambda_{r,r+1} + x_{r+2} \lambda_{r,r+2} + \ldots + x_{n} \lambda_{r, n} \tag{4}
    \end{cases}
  \end{gather*}
  Т.к. $X^{(1)}, X^{(2)}, \ldots X^{(k)}$ образуют ФСР, то они удовлетворяют системе (4):
  \begin{gather*}
    \begin{cases}
      X_1^{(i)} = X_{r+1}^{(i)} \lambda_{1,r+1} + X_{r+2}^{(i)} \lambda_{1,r+2} + \ldots + X_{n}^{(i)} \lambda_{1, n} \\
      X_2^{(i)} = X_{r+1}^{(i)} \lambda_{2,r+1} + X_{r+2}^{(i)} \lambda_{2,r+2} + \ldots + X_{n}^{(i)} \lambda_{2, n} \\
      \ldots \\
      X_r^{(i)} = X_{r+1}^{(i)} \lambda_{n,r+1} + X_{r+2}^{(i)} \lambda_{n,r+2} + \ldots + X_{n}^{(i)} \lambda_{n, n} \tag{5}
    \end{cases} \quad 
    X^{(i)} = 
    \begin{pmatrix}
      x_1^{(i)} \\ x_2^{(i)} \\ \ldots \\ x_m^{(i)}
    \end{pmatrix}
  \end{gather*}

  Составим матрицу $B$ из столбцов $X^{(i)}$ :
  \begin{align*}
    B =
    \begin{pmatrix}
      x_1 & x_1^{(1)} & x_1^{(2)} & \ldots & x_1^{(n)} \\
      x_2 & x_2^{(1)} & x_2^{(2)} & \ldots & x_2^{(n)} \\
      \ldots & \ldots & \ldots & \ldots & \ldots \\
      x_r & x_r^{(1)} & x_r^{(2)} & \ldots & x_r^{(n)} \\
      \ldots & \ldots & \ldots & \ldots & \ldots \\
      x_n & x_n^{(1)} & x_n^{(2)} & \ldots & x_n^{(n)} \\
    \end{pmatrix} \\
    \begin{matrix}
      X & X^{(1)} & X^{(2)} & \ldots & X^{(n)}
    \end{matrix}
  \end{align*} 

  Вычтем из элементов первой строки соответствующие элементы строк с $r+1$ по $m$ с соответствующим коеффициентом  $\lambda_{1,r+1}, \lambda_{1,r+2}, \ldots \lambda_{1n}$:
  \begin{gather*}
    \begin{cases}
      x_1^{(1)} - \lambda_{1,r+1} x_{r+1}^{(1)} - \lambda_{1,r+2} x_{r+2}^{(2)} - \ldots - \lambda_{1,n} x_{n}^{(1)} = 0 \\
      x_1^{(2)} - \lambda_{1,r+1} x_{r+1}^{(2)} - \lambda_{1,r+2} x_{r+2}^{(2)} - \ldots - \lambda_{1,n} x_{n}^{(2)} = 0 \\
      \ldots \\
      x_1^{(k)} - \lambda_{1,r+1} x_{r+1}^{(k)} - \lambda_{1,r+2} x_{r+2}^{(k)} - \ldots - \lambda_{1,n} x_{n}^{(k)} = 0 \\
      x_1 - \lambda_{1,r+1} x_{r+1} - \lambda_{1,r+2} x_{r+2} - \ldots - \lambda_{1,n} x_{n} = 0
    \end{cases}
  \end{gather*}

  Аналогично вычитая из строк до r строки $r+1$ до $n$ с коеффициентами  $\lambda$.
  В результате получаем, что в преобразованной матрице $B$ первые  $r$ строк будут нулевые:
  
  \begin{gather*}
    B =
    \begin{pmatrix}
      0 & 0 & 0 & \ldots & 0 \\
      0 & 0 & 0 & \ldots & 0 \\
      \ldots & \ldots & \ldots & \ldots & \ldots \\
      0 & 0 & 0 & \ldots & 0 \\
      x_{r+1} & x_{r+1}^{(1)} & x_{r+1}^{(2)} & \ldots & x_{r+1}^{(n)} \\
      \ldots & \ldots & \ldots & \ldots & \ldots \\
      x_n & x_n^{(1)} & x_n^{(2)} & \ldots & x_n^{(n)} \\
    \end{pmatrix}
  \end{gather*} 

  Поскольку элементарные преобразования не меняют ранга матрицы, получаем, что ранг матрицы $B$ будет равен $k = n - r$.
  Так как по условию столбцы $X^{(1)}, X^{(2)}, \ldots X^{(k)}$ образуют ФСР, они являются линейно-независимыми. Поэтому первый столбец можно представить в виде линейной комбинации столбцов.

\end{answer}

